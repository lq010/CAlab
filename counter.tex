\section{Project 4 - Counter}

To design the counter we choose to do it with flip flop T type with and gate (see figure ???) known also as "frequency divider".

\begin{figure}
    \centering
    \includegraphics[width=3.0in]{counter_structure}
    \caption{Structure of the counter}
    \label{counterstructure}
\end{figure}

In this implementation a number of T flip flop (equal to the number of bit of the counter) are connected so that "toggles" only when the previous one has output and input high. In this way they toggle sequentially one "toggle period" later of the previous flip-flop, so follows the behavior of the binary numbers in which each bit changes from 0 to 1 in function of its position, when looking at sequential integer number written one below another (see figure ???). In this way if we take the output Q of each flip-flop we can see it as a string of bit that represents an integer number increasing each clock cycle.

\begin{tabular}{ l | c | r }
  Q0 & Q1 & Q2 \\
  0 & 0 & 0 \\
  0 & 0 & 1 \\
  0 & 1 & 0 \\
  0 & 1 & 1 \\
  1 & 0 & 0 \\
  1 & 0 & 1 \\
  1 & 1 & 0 \\
  1 & 1 & 1 \\
\end{tabular}

This behavior is obtained connecting the input and output of each flip-flop to an and gate that propagates the signal to next flip-flop input, but also to the next and gate. For this reason the "and gate chain" is the critical path of the circuit. For example in this 32 bit counter circuit we will have 30 and gate connected in series (see figure ???), that propagate the first input to the input of the last T flip flop. So the clock frequency is limited by this path.

\begin{figure}
    \centering
    \includegraphics[width=3.0in]{and_chain}
    \caption{and gates forming critical path}
    \label{andchain}
\end{figure}

The maximum clock frequency is also limited by the delay introduced by interconnections that binds the mapped LEs output to the external pin that connects the FPGA to the LEDs or 7 segment display. The position of peripheral resources are fixed during the board design and we can't modify them. Because of that, is difficult to optimize the circuit if not choosing different counter architecture. (see figure ???)

\begin{figure}
    \centering
    \includegraphics[width=3.0in]{counter_pinplan}
    \caption{Pin position: notice CLK (left side) is opposite to the LED (bottom right)}
    \label{counterpinplan}
\end{figure}

Because we connect the CLK signal to the 50MHz oscillator of the board, we set a maximum delay constraint of tp=20ns for the combinational "and chain". The timing analysis show that this requirement is met with some slack:

(table of setup, hold time, max frequency allowed)
 